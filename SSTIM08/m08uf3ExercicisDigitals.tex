\documentclass[a4paper]{article}

\usepackage[catalan]{babel}
\usepackage[utf8]{inputenc}
\usepackage{amsmath}
\usepackage{graphicx}
\usepackage[colorinlistoftodos]{todonotes}
\usepackage{circuitikz}

\title{Problemes de Lògica Combinacional}

\author{Departament d' Electrònica}

\date{\today}

\begin{document}
\maketitle

\begin{abstract}
Problemes de Lògica Combinacional corresponents a la tercera unitat formativa del mòdul M08 Elements de sistemes de telecomunicacions.
\end{abstract}

\section{Introducció}

\subsection{Troba la funció resultant a partir dels següents circuits}

\begin{circuitikz} \draw
(0,2) node[and port] (myand) {}
(2,1) node[or port] (myor) {}
(myand.in 1) node[above left=.5cm](a) {A}
(myand.in 2) node[below left = .5cm](b) {B}
(myand.out) -| (myor.in 1)
(a) -| (myand.in 1)
(b) -| (myand.in 2)
(b) node[below=1cm](c){C}
(c) -| (myor.in 2);
\end{circuitikz}

\begin{circuitikz} \draw
(0,2) node[or port] (myor) {}
(2,1) node[or port] (myor) {}
(myand.in 1) node[above left=.5cm](a) {A}
(myand.in 2) node[below left = .5cm](b) {B}
(myand.out) -| (myor.in 1)
(a) -| (myand.in 1)
(b) -| (myand.in 2)
(b) node[below=1cm](c){C}
(c) -| (myor.in 2);
\end{circuitikz}

\begin{circuitikz} \draw
(0,2) node[and port] (myand) {}
(2,1) node[or port] (myor) {}
(myand.in 1) node[above left=.5cm](a) {A}
(myand.in 2) node[below left = .5cm](b) {B}
(myand.out) -| (myor.in 1)
(a) -| (myand.in 1)
(b) -| (myand.in 2)
(b) node[below=1cm](c){C}
(c) -| (myor.in 2);
\end{circuitikz}

\section{Funcions i Circuits}


\subsection{Lleis de De Morgan}
Les lleis de De Morgan declaren que la suma de n variables globalment negades (o invertides) és igual al producte de les n variables negades individualment, i que inversament, el producte de n variables globalment negades és igual a la suma de les n variables negades individualment.

Utilitza les Lleis de De Morgan per simplificar les següents funcions

\subsection{Sumador i restador}

Realitza un sumador de dos bits

\begin{table}[!htbp]
\centering
\begin{tabular}{ll|rr}
x&y&C&S \\\hline
0&0&0&0 \\
0&1&0&1 \\
1&0&0&1 \\
1&1&1&0 \\

\end{tabular}
\caption{\label{tab:widgets}Taula Veritat Sumador de 2 bits.}
\end{table}

Realitza un sumador de tres bits

\begin{table}[!htbp]
\centering
\begin{tabular}{lll|rr}
x&y&z&C&S \\\hline
0&0&0&0&0 \\
0&0&1&0&1 \\
0&1&0&0&1 \\
0&1&1&1&0 \\
1&0&0&0&1 \\
1&0&1&1&0 \\
1&1&0&1&0 \\
1&1&1&1&1 \\

\end{tabular}
\caption{\label{tab:widgets}Taula Veritat Sumador de 3 bits.}
\end{table}

\section{Karnaugh}
\label{sec:introduction}

Els mapes de Karnaug permeten trobar la funci\'o algebraica booleana partint de la taula de veritat.

\subsection{Display de 7 elements}
Dissenya el circuit combinacional que faci funcionar un display digital de 7 segments.
\begin{itemize}
\item El display mostrar\`a els d\'igits del 0 al 9.
\item L'entrada ser\`a binaria de 4 bits amb els valor en binari de 0 a 9.
\item El circuit resultant tindr\`a 4 entrades i 7 sortides.
\item Les entrades del 10 al 15 no es donaran mai. 
\item Construeix el circuit amb el simulador LOGO! Soft Comfort.
\end{itemize}

\section{Multiplexor i Demultiplexor}
El Muliplexor permet incloure múltiples línes en una única. Permet per exemple codificar n entrades en una única sortida digital.

\begin{description}
\item[Exemple: MUX 4x1] Permet consultar l'estat de 4 entrades binaries seleccionant una amb una entrada de dos bits.
\end{description}

\begin{table}[!htbp]
\centering
\begin{tabular}{l l l l|c c|r}
I_{0} & I_{1} & I_{2} & I_{3} & S_{1} & S_{2} & S \\\hline
1&0&0&0&0&0&I_{0} \\
0&1&0&0&0&1&I_{1} \\
0&0&1&0&1&0&I_{2} \\
0&0&0&1&1&1&I_{3} \\

\end{tabular}
\caption{\label{tab:widgets}Taula Veritat MUX4x1.}
\end{table}

\subsection{Hall Effect}
Explain the classical Hall effect in your own words. What do I measure at $B=0$? And what happens if $B>0$? Which effect gives rise to the voltage drop in the vertical direction?

\subsection{Quantum Hall Effect}
Explain the IQHE in your own words. What does the density of states look like in a 2-DEG when $B=0$? What are Landau levels and how do they arise? What are edge states? What does the electron transport look like when you change the magnetic field? What do you expect to measure?

\section{Experiment 1-2 pages}
\subsection{Fabrication}
Explain a step-by-step recipe for fabrication here. How long did you etch and why? What is an Ohmic contact?
\subsection{Experimental set-up}
Explain the experimental set-up here. Use a schematic picture (make it yourself in photoshop, paint, ...) to show how the components are connected. Briefly explain how a lock-in amplifier works.

\section{Results and interpretation 2-3 pages}
Show a graph of the longitudinal resistivity ($\rho_{xx}$) and Hall resistivity ($\rho_{xy}$) versus magnetic field, extracted from the raw data shown in figure \ref{fig:data}. You will have the link to the data in your absalon messages, if not e-mail Guen (guen@nbi.dk). Explain how you calculated these values, and refer to the theory.


\subsection{Classical regime}
Calculate the sheet electron density $n_{s}$ and electron mobility $\mu$ from the data in the low-field regime, and refer to the theory in section \ref{sec:theory}. Explain how you retrieved the values from the data (did you use a linear fit?).
Round values off to 1 or 2 significant digits: 8.1643 ~= 8.2. Also, 5e-6 is easier to read than 0.000005.

!OBS: This part is optional (only if you have time left).
Calculate the uncertainty as follows: \newline $u(f(x, y, z)) = \sqrt{(\frac{\delta f}{\delta{x}} u(x))^{2} + (\frac{\delta f}{\delta{y}} u(y))^{2} + (\frac{\delta f}{\delta{z}} u(z))^{2}}$, where $f$ is the calculated value ($n_{s}$ or $\mu$), $x, y, z$ are the variables taken from the measurement and $u(x)$ is the uncertainty in x (and so on).

\subsection{Quantum regime}
Calculate $n_{s}$ for the high-field regime.
Show a graph of the longitudinal conductivity ($\rho_{xx}$) and Hall conductivity($\rho_{xy}$) \textbf{in units of the resistance quantum} ($\frac{h}{e^{2}}$), depicting the integer filling factors for each plateau.
Show a graph of the plateau number versus its corresponding value of $1/B$. From this you can determine the slope, which you use to calculate the electron density.
Again, calculate the uncertainty for your obtained values.

\section{Discussion 1/2-1 page}
Discuss your results. Compare the two values of $n_{s}$ that you've found in the previous section. Compare your results with literature and comment on the difference. If you didn't know the value of the resistance quantum, would you be able to deduce it from your measurements? If yes/no, why?

\newpage
\section{Some LaTeX tips}
\label{sec:latex}
\subsection{How to Include Figures}

First you have to upload the image file (JPEG, PNG or PDF) from your computer to writeLaTeX using the upload link the project menu. Then use the includegraphics command to include it in your document. Use the figure environment and the caption command to add a number and a caption to your figure. See the code for Figure \ref{fig:frog} in this section for an example.



\subsection{How to Make Tables}

Use the table and tabular commands for basic tables --- see Table~\ref{tab:widgets}, for example.

\begin{table}
\centering
\begin{tabular}{l|r}
Item & Quantity \\\hline
Widgets & 42 \\
Gadgets & 13
\end{tabular}
\caption{\label{tab:widgets}An example table.}
\end{table}

\subsection{How to Write Mathematics}

\LaTeX{} is great at typesetting mathematics. Let $X_1, X_2, \ldots, X_n$ be a sequence of independent and identically distributed random variables with $\text{E}[X_i] = \mu$ and $\text{Var}[X_i] = \sigma^2 < \infty$, and let

\begin{equation}
S_n = \frac{X_1 + X_2 + \cdots + X_n}{n}
      = \frac{1}{n}\sum_{i}^{n} X_i
\label{eq:sn}
\end{equation}

denote their mean. Then as $n$ approaches infinity, the random variables $\sqrt{n}(S_n - \mu)$ converge in distribution to a normal $\mathcal{N}(0, \sigma^2)$.

The equation \ref{eq:sn} is very nice.

\subsection{How to Make Sections and Subsections}

Use section and subsection commands to organize your document. \LaTeX{} handles all the formatting and numbering automatically. Use ref and label commands for cross-references.

\subsection{How to Make Lists}

You can make lists with automatic numbering \dots

\begin{enumerate}
\item Like this,
\item and like this.
\end{enumerate}
\dots or bullet points \dots
\begin{itemize}
\item Like this,
\item and like this.
\end{itemize}
\dots or with words and descriptions \dots
\begin{description}
\item[Word] Definition
\item[Concept] Explanation
\item[Idea] Text
\end{description}

We hope you find write\LaTeX\ useful, and please let us know if you have any feedback using the help menu above.

\begin{thebibliography}{9}
\bibitem{nano3}
  K. Grove-Rasmussen og Jesper Nygård,
  \emph{Kvantefænomener i Nanosystemer}.
  Niels Bohr Institute \& Nano-Science Center, Københavns Universitet

\end{thebibliography}
\end{document}